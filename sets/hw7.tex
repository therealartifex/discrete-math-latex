\begin{enumerate}[leftmargin=2cm,labelsep=.5cm,label=\bf\arabic*.]
\item

\item
\begin{lemma}
$\sqrt{3}$ is an irrational number.
\end{lemma}
\begin{proof}
Assume that $\sqrt{3}$ is a rational number.\\
If $n^2 \bmod k = 0$, then $n \bmod k = 0$.\\
A rational number can be expressed as the quotient $\frac{p}{q}$ of two integers, where $q\neq 0$.\\[2mm]
$\sqrt{3}=\frac{p}{q}$, where $p$ and $q$ are coprime integers.\\
$3q^2 = p^2$.\\
$p^2$ is divisible by $3$, so $p$ is also divisible by $3$.\\
Let $p=3n$ for some integer n.\\
Then $p \bmod 3 = 0$.\\
$3q^2 = (3n)^2$\\
$3q^2 = 9n^2$\\
$q^2 = 3n^2$.\\
If $3n^2 \bmod 3 = 0$, then $q^2 \bmod 3 = 0$.\\[2mm]
Now $q^2$ is divisible by $3$, so $q$ must also be divisible by $3$.\\
If both $p$ and $q$ are divisible by $3$, then they are not coprime.\\
Therefore, $\sqrt{3}$ must be an irrational number.\\[5mm]
\end{proof}

\item
\begin{lemma}
A triangle cannot have more than one obtuse angle.
\end{lemma}
\begin{proof}
Assume that a triangle can have more than one obtuse angle.\\
Any angle greater than $90^{\circ}$ and less than $180^{\circ}$ is obtuse.\\
A triangles angles must add up to $180^{\circ}$.\\[2mm]
Let $a$, $b$, $c$ be the angles of a triangle.\\
Then $a + b + c = 180$.\\
Let $a > 90$, and $b > 90$.\\[2mm]
Then $a + b > 180$, which dissatisfies the definition of a triangle.\\
Therefore, a triangle cannot have more than one obtuse angle.\\[5mm]
\end{proof}

\item
\begin{enumerate}
  \item $\big\{\varnothing,\left\{0\right\},\left\{4\right\},\left\{0,4\right\} \big\}$
  \item $4$
  \item $\left\{4,1,2 \right\}$
  \item $\left\{0,3,4,5 \right\}$
  \item $\left\{0,1,4 \right\}$
  \item $\left\{4 \right\}$
  \item $\left\{2 \right\}$
  \item $\big\{\left(0,0\right),\left(0,1\right),\left(0,4\right),\left(0,5\right),\left(4,0\right),\left(4,1\right),\left(4,4\right),\left(4,5\right) \big\}$
  \item $8$
  \item $64$
  \item $\big\{\left(0,0,3,3\right),\left(0,1,3,3\right),\left(0,2,3,3\right),\left(4,0,3,3\right),\left(4,1,3,3\right),\left(4,2,3,3\right) \big\}$
  \item $\left\{3,3,3,3,3,3,3,3 \right\}$
  \item $6$
  \item $8$\\[5mm]
\end{enumerate}

\item
\begin{tabu}[t]{|c|c|c|c|}
\hline
$A$ & $B$ & $A\cup B$ & $(A\cup B)\Delta B$ \\ \hline
\textbf{1} & \textbf{1} & \textbf{1} & \textbf{0} \\ \hline
\textbf{1} & \textbf{1} & \textbf{1} & \textbf{0} \\ \hline
\textbf{1} & \textbf{0} & \textbf{1} & \textbf{1} \\ \hline
\textbf{1} & \textbf{0} & \textbf{1} & \textbf{1} \\ \hline
\textbf{0} & \textbf{1} & \textbf{1} & \textbf{0} \\ \hline
\textbf{0} & \textbf{1} & \textbf{1} & \textbf{0} \\ \hline
\textbf{0} & \textbf{0} & \textbf{0} & \textbf{0} \\ \hline
\textbf{0} & \textbf{0} & \textbf{0} & \textbf{0} \\ \hline
\end{tabu}\\[5mm]

\item $
\begin{aligned}[t]
(A \cap \overline{B}) \cup (\overline{A} \cup B) &= (\overline{B} \cap A) \cup (\overline{A} \cup B) & \text{[Commutative]}\\
&= \overline{B} \cap (A \cup \overline{A}) \cup B & \text{[Associative]}\\
&= \overline{B} \cap \mathscr{E} \cup B & \text{[Inverse]}\\
&= \overline{B} \cup B & \text{[Identity]}\\
&= \mathscr{E} & \text{[Inverse]}
\end{aligned} $
\end{enumerate}